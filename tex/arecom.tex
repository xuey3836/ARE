%%%%%%%%%%%%%%%%%%%%%%%%%%%%%%%%%%%%%%%%%
% Lachaise Assignment
% LaTeX Template
% Version 1.0 (26/6/2018)
%
% This template originates from:
% http://www.LaTeXTemplates.com
%
% Authors:
% Marion Lachaise & François Févotte
% Vel (vel@LaTeXTemplates.com)
%
% License:
% CC BY-NC-SA 3.0 (http://creativecommons.org/licenses/by-nc-sa/3.0/)
%
%%%%%%%%%%%%%%%%%%%%%%%%%%%%%%%%%%%%%%%%%

%----------------------------------------------------------------------------------------
%	PACKAGES AND OTHER DOCUMENT CONFIGURATIONS
%----------------------------------------------------------------------------------------

\documentclass{article}
%%%%%%%%%%%%%%%%%%%%%%%%%%%%%%%%%%%%%%%%%
% Lachaise Assignment
% Structure Specification File
% Version 1.0 (26/6/2018)
%
% This template originates from:
% http://www.LaTeXTemplates.com
%
% Authors:
% Marion Lachaise & François Févotte
% Vel (vel@LaTeXTemplates.com)
%
% License:
% CC BY-NC-SA 3.0 (http://creativecommons.org/licenses/by-nc-sa/3.0/)
% 
%%%%%%%%%%%%%%%%%%%%%%%%%%%%%%%%%%%%%%%%%

%----------------------------------------------------------------------------------------
%	PACKAGES AND OTHER DOCUMENT CONFIGURATIONS
%----------------------------------------------------------------------------------------

\usepackage{amsmath,amsfonts,stmaryrd,amssymb} % Math packages

\usepackage{enumerate} % Custom item numbers for enumerations

\usepackage[ruled]{algorithm2e} % Algorithms

\usepackage[framemethod=tikz]{mdframed} % Allows defining custom boxed/framed environments

\usepackage{listings} % File listings, with syntax highlighting
\lstset{
	basicstyle=\ttfamily, % Typeset listings in monospace font
}

%----------------------------------------------------------------------------------------
%	DOCUMENT MARGINS
%----------------------------------------------------------------------------------------
\usepackage{XCharter}

\usepackage{geometry} % Required for adjusting page dimensions and margins

\geometry{
	paper=a4paper, % Paper size, change to letterpaper for US letter size
	top=2.5cm, % Top margin
	bottom=3cm, % Bottom margin
	left=2.5cm, % Left margin
	right=2.5cm, % Right margin
	headheight=14pt, % Header height
	footskip=1.5cm, % Space from the bottom margin to the baseline of the footer
	headsep=1.2cm, % Space from the top margin to the baseline of the header
	%showframe, % Uncomment to show how the type block is set on the page
}

%----------------------------------------------------------------------------------------
%	FONTS
%----------------------------------------------------------------------------------------

\usepackage[utf8]{inputenc} % Required for inputting international characters
\usepackage[T1]{fontenc} % Output font encoding for international characters

\usepackage{XCharter} % Use the XCharter fonts

%----------------------------------------------------------------------------------------
%	COMMAND LINE ENVIRONMENT
%----------------------------------------------------------------------------------------

% Usage:
% \begin{commandline}
%	\begin{verbatim}
%		$ ls
%		
%		Applications	Desktop	...
%	\end{verbatim}
% \end{commandline}

\mdfdefinestyle{commandline}{
	leftmargin=10pt,
	rightmargin=10pt,
	innerleftmargin=15pt,
	middlelinecolor=black!50!white,
	middlelinewidth=2pt,
	frametitlerule=false,
	backgroundcolor=black!5!white,
	frametitle={Command Line},
	frametitlefont={\normalfont\sffamily\color{white}\hspace{-1em}},
	frametitlebackgroundcolor=black!50!white,
	nobreak,
}

% Define a custom environment for command-line snapshots
\newenvironment{commandline}{
	\medskip
	\begin{mdframed}[style=commandline]
}{
	\end{mdframed}
	\medskip
}

%----------------------------------------------------------------------------------------
%	FILE CONTENTS ENVIRONMENT
%----------------------------------------------------------------------------------------

% Usage:
% \begin{file}[optional filename, defaults to "File"]
%	File contents, for example, with a listings environment
% \end{file}

\mdfdefinestyle{file}{
	innertopmargin=1.6\baselineskip,
	innerbottommargin=0.8\baselineskip,
	topline=false, bottomline=false,
	leftline=false, rightline=false,
	leftmargin=2cm,
	rightmargin=2cm,
	singleextra={%
		\draw[fill=black!10!white](P)++(0,-1.2em)rectangle(P-|O);
		\node[anchor=north west]
		at(P-|O){\ttfamily\mdfilename};
		%
		\def\l{3em}
		\draw(O-|P)++(-\l,0)--++(\l,\l)--(P)--(P-|O)--(O)--cycle;
		\draw(O-|P)++(-\l,0)--++(0,\l)--++(\l,0);
	},
	nobreak,
}

% Define a custom environment for file contents
\newenvironment{file}[1][File]{ % Set the default filename to "File"
	\medskip
	\newcommand{\mdfilename}{#1}
	\begin{mdframed}[style=file]
}{
	\end{mdframed}
	\medskip
}

%----------------------------------------------------------------------------------------
%	NUMBERED QUESTIONS ENVIRONMENT
%----------------------------------------------------------------------------------------

% Usage:
% \begin{question}[optional title]
%	Question contents
% \end{question}

\mdfdefinestyle{question}{
	innertopmargin=1.2\baselineskip,
	innerbottommargin=0.8\baselineskip,
	roundcorner=5pt,
	nobreak,
	singleextra={%
		\draw(P-|O)node[xshift=1em,anchor=west,fill=white,draw,rounded corners=5pt]{%
		Question \theQuestion\questionTitle};
	},
}

\newcounter{Question} % Stores the current question number that gets iterated with each new question

% Define a custom environment for numbered questions
\newenvironment{question}[1][\unskip]{
	\bigskip
	\stepcounter{Question}
	\newcommand{\questionTitle}{~#1}
	\begin{mdframed}[style=question]
}{
	\end{mdframed}
	\medskip
}

%----------------------------------------------------------------------------------------
%	WARNING TEXT ENVIRONMENT
%----------------------------------------------------------------------------------------

% Usage:
% \begin{warn}[optional title, defaults to "Warning:"]
%	Contents
% \end{warn}

\mdfdefinestyle{warning}{
	topline=false, bottomline=false,
	leftline=false, rightline=false,
	nobreak,
	singleextra={%
		\draw(P-|O)++(-0.5em,0)node(tmp1){};
		\draw(P-|O)++(0.5em,0)node(tmp2){};
		\fill[black,rotate around={45:(P-|O)}](tmp1)rectangle(tmp2);
		\node at(P-|O){\color{white}\scriptsize\bf !};
		\draw[very thick](P-|O)++(0,-1em)--(O);%--(O-|P);
	}
}

% Define a custom environment for warning text
\newenvironment{warn}[1][Warning:]{ % Set the default warning to "Warning:"
	\medskip
	\begin{mdframed}[style=warning]
		\noindent{\textbf{#1}}
}{
	\end{mdframed}
}

%----------------------------------------------------------------------------------------
%	INFORMATION ENVIRONMENT
%----------------------------------------------------------------------------------------

% Usage:
% \begin{info}[optional title, defaults to "Info:"]
% 	contents
% 	\end{info}

\mdfdefinestyle{info}{%
	topline=false, bottomline=false,
	leftline=false, rightline=false,
	nobreak,
	singleextra={%
		\fill[black](P-|O)circle[radius=0.4em];
		\node at(P-|O){\color{white}\scriptsize\bf i};
		\draw[very thick](P-|O)++(0,-0.8em)--(O);%--(O-|P);
	}
}

% Define a custom environment for information
\newenvironment{info}[1][Info:]{ % Set the default title to "Info:"
	\medskip
	\begin{mdframed}[style=info]
		\noindent{\textbf{#1}}
}{
	\end{mdframed}
}
 % Include the file specifying the document structure and custom commands

%----------------------------------------------------------------------------------------
%	ASSIGNMENT INFORMATION
%----------------------------------------------------------------------------------------

\title{The computation of four AREs} % Title of the assignment

\author{Yuan Xue\\ \texttt{xueyuan115@mails.ucas.ac.cn}} % Author name and email address

\date{\today} % University, school and/or department name(s) and a date

%----------------------------------------------------------------------------------------
\usepackage{booktabs}
\begin{document}

\maketitle % Print the title

%----------------------------------------------------------------------------------------
%	INTRODUCTION
%----------------------------------------------------------------------------------------

\section{Introduction} % Unnumbered section


\subsection{Notation}
likelihood function

\begin{equation}
 L_n(\lambda_1,\lambda_2,\eta) = \frac{\lambda_1^{r_1}\lambda_2^{r2}exp(r\eta)}{\{1+exp(\eta)\}^{n_0}\{1+\lambda_1exp(\eta)\}^{n_1}\{1+\lambda_2exp(\eta)\}^{n_2}}
\end{equation}

log-likelihood function $l_n(\lambda,\eta,\theta)$

we work with $l_n(\lambda,1-\theta-+\theta\lambda,\eta)$, where $\theta$  is explicitly epressed.

\begin{align*}
x_1 = & \lambda\\
x_2= & 1-\theta-\theta\lambda\\
x_3 = & \eta
\end{align*}

Denote

\begin{align*}
  l_{n,\mu} =& \partial l_n/\partial x_\mu \quad \text{for} \quad \mu= 1,2,3 \\
  l_{n,\mu\nu}= & \partial^2 l_n/\partial x_\mu \partial x_\nu \quad \text{for} \quad \mu=1,2, \nu= 1,2,3\\
  l_{n,33}= & \partial^2 l_n/\partial x_3 \partial x_3\\
  l_{n,\mu\nu} = & l_{n,\nu\mu} \quad \text{for} \quad \mu,\nu = 1,2\\
  l_{n,\mu\nu} = & l^T_{n,\nu\mu} \quad \text{for} \quad \mu= 1,2, \nu=3\\
  L_{\mu\nu}(\eta) =& E_{H_0}(l_{1,\mu\nu}(1,1,\eta))\quad \text{for} \quad \mu=1,2,3, \nu= 1,2,3\\
\end{align*}
$$s(\theta,\eta) = l_{1,1}(1,1,\eta)+\theta l_{1,2}(1,1,\eta)-(L_{13}^T(\eta)+\theta L_{23}^T(\eta))L_{33}^{-1}(\eta)l_{1,3}(1,1,\eta)$$\\
\subsection{Algorithm}
\begin{itemize}
\item Input: $Y$, $G$, $\theta^{(0)}$, $\theta_i$, $\theta_j$
\item Output: $e_P(Z_{MERT},Z_{\theta^{(0)}})$, $\tilde{e}_C(Z_{MERT},Z_{\theta^{(0)}})$, $e_{HL}(Z_{MERT},Z_{\theta^{(0)}})$, $e_{B}(Z_{MERT},Z_{\theta^{(0)}})$
\end{itemize}

\begin{enumerate}[step 1]
\item Estimate $\hat{\eta}$. where $\hat{\eta}$ satisfy $\partial l_n/\partial \eta|_{H_0,\hat{\eta}_n}= l_{n,3}(1,1,\hat{\eta}_n)=0$.
\item Compute $l_n(1,1,\hat{\eta})$; $l_{n,\mu}(1,1,\hat{\eta})$, for $\mu= 1,2,3$; $l_{n,\mu\nu}(1,1,\hat{\eta})$ for $\mu= 1,2,3$, $\nu=1,2,3$.
\item Compute $L_{\mu\nu}(\hat{\eta}) = E_{H_0}(l_{1,{\mu\nu}}(1,1,\hat{\eta})) ={\color{red} \frac{1}{n}l_{n,\mu\nu}(1,1,\hat{\eta}) }$
\item Compute $\sigma(\theta^{(0)})$, $\sigma(\theta_i)$, $\sigma(\theta_j), \sigma(\theta^{(0)},\theta_i), \sigma(\theta^{(0)},\theta_j), \sigma(\theta_i,\theta_j)$, where $$\sigma(\theta_i,\theta_j) =A_{\hat{\eta}}\theta_i\theta_j+B_{\hat{\eta}}(\theta_i+\theta_j)+C_{\hat{\eta}}$$
\begin{align*}
 A_\eta =& L_{23}(\eta)L_{33}^{-1}(\eta)L_{32}(\eta)-L_{22}(\eta)\\
 B_\eta =& L_{13}(\eta)L_{33}^{-1}(\eta)L_{31}(\eta)-L_{12}(\eta)\\
 C_\eta =& L_{13}(\eta)L_{33}^{-1}(\eta)L_{31}(\eta)-L_{11}(\eta)
\end{align*}
\item Compute $\mu(\lambda,\theta^{(0)}),\mu(\lambda,\theta_i),\mu(\lambda,\theta_j)$,
$$\mu(\lambda,\theta)= E_{H_1,\eta_0}(s(\theta,\eta_{\theta}))= {\color{red} \frac{1}{n}(l_{n,11}(n,1,\hat{\eta})+\theta l_{n,21}(n,1,\hat{\eta})- (L_{13}^T(\hat{\eta})+\theta L_{23}^T(\hat{\eta}))L_{33}^{-1}(\hat{\eta})l_{n,31}(1,1,\hat{\eta}))}$$
\item Compute $\mu^{(1)}(\lambda,\theta^{(0)}),\mu^{(1)}(\lambda,\theta_i),\mu^{(1)}(\lambda,\theta_j)$,

\begin{align*}
\mu^{(1)}(\lambda,\theta)=& {\color{red} \frac{1}{n}(l_{n,11}(n,1,\hat{\eta})+\theta l_{n,21}(n,1,\hat{\eta})- (L_{13}^T(\hat{\eta})+\theta L_{23}^T(\hat{\eta}))L_{33}^{-1}(\hat{\eta})l_{n,31}(1,1,\eta))}\\
= &{\color{red} L_{11}(\hat{\eta})+\theta L_{21}(\hat{\eta})- (L_{13}^T(\hat{\eta})+\theta L_{23}^T(\hat{\eta}))L_{33}^{-1}(\hat{\eta})L_{31}(\hat{\eta}))}
\end{align*}

\item Compute $e_P(Z_{MERT},Z(\theta^{(0)}))$. $$e_P(Z_{MERT},Z(\theta^{(0)}))=\frac{(\rho_{\theta_i,\theta^{(0)}}+\rho_{\theta_j,\theta^{(0)}})^2}{2(1+\rho_{\theta_i,\theta_j})}$$

$$\rho_{\theta_i,\theta_j}= \frac{\sigma(\hat{\eta},\theta_i,\theta_j)}{{\sigma(\hat{\eta},\theta_i,\theta_i)\sigma(\hat{\eta},\theta_i,\theta_j)}^{1/2}}$$

\item Compute \begin{align*}
\tilde{e}_C(Z_{MERT},Z_{\theta^{(0)}})= \frac{ \tilde{Q}_{Z_{MERT}}}{\tilde{Q}_{Z_{\theta^{(0)}}}}\\
\tilde{Q}_{Z_{\theta^{(0)}}} =  2\left(1-\Phi\left(\frac{\mu^{(1)}(\lambda_0,\theta^{(0)})}{2\sigma(\theta^{(0)})}\right)\right)\\
\tilde{Q}_{Z_{MERT}} =  2\left(1-\Phi\left(\left[\frac{\mu^{(1)}(\lambda_0,\theta_i)}{2\sigma(\theta_i)}+\frac{\mu^{(1)}(\lambda_0,\theta_j)}{2\sigma(\theta_j)}\right]/\sqrt{8(1+\rho_{\theta_i,\theta_j})}\right)\right)\\
\color{red} \lambda_0 = 1
\end{align*}

\item Compute  \begin{align*}
 e_{HL}(Z_{MERT},Z_{\theta}) =\frac{d_{Z_{MERT}}(\lambda)}{d_{Z_{\theta}(\lambda)}}\\
 d_{Z_\theta}(\lambda) =  \frac{\mu^2(\lambda,\theta)}{\sigma^2(\theta)}\\
 d_{Z_{MERT}}(\lambda) = \mu^2_{MERT}(\lambda)\\
 \mu_{MERT}(\lambda)=[\mu(\lambda,\theta_i)/\sigma(\theta_i)+\mu(\lambda,\theta_j)/\sigma(\theta_j)]/\sqrt{2(1+\rho_{\theta_i,\theta_j})}
\end{align*}

\item Compute  \begin{align*}
 e_{B}(Z_{MERT},Z_{\theta}) = e_{HL}(Z_{MERT},Z_{\theta})\\
 \tilde{e}_{B}(Z_{MERT},Z_{\theta}) = \frac{\tilde{c}_{Z_{MERT}}}{\tilde{c}_{Z_{\theta}}}\\
 \tilde{c}_{Z_{MERT}} = 1 - \Phi(\mu^{(1)}_{MERT}(1))\\
 \tilde{c}_{Z_{\theta}} = 1 - \Phi(\mu^{(1)}(1,\theta^{(0)})/\sigma(\theta^{(0)}))\\
 \mu^{(1)}_{MERT}(\lambda))=[\mu^{(1)}(\lambda,\theta_i)/\sigma(\theta_i)+\mu^{(1)}(\lambda,\theta_j)/\sigma(\theta_j)]/\sqrt{2(1+\rho_{\theta_i,\theta_j})}
\end{align*}
\end{enumerate}

\subsection{simulation}
Let $p$ be the minor allele frequency (MAF) of the marker of interest in the population. we consider case-control data with $r=500$ cases and $s= 500$ controls. and $\lambda \in \{1.1, 1.2, 1.3\}$ and $p \in \{0.15,0.30, 0.45 \}$, and the true $\theta^{(0)} \in \{0,1/4,1/2,1\}$. . we generate $Nrep=1000$ datasets. and we compute the mean and variance of the four AREs to  $Z_{MERT}$  and $Z_{\theta^{(0)}}$


% Please add the following required packages to your document preamble:

\begin{table}[]
\centering
\caption{The Four AREs of $Z_{MERT}$ and $Z_{\theta^{(0)}}$}
\begin{tabular}{@{}cccccccccccccccc@{}}
\toprule
     &                       & \multicolumn{4}{c}{$\lambda=1.1$} &  & \multicolumn{4}{c}{$\lambda=1.3$} &  &\multicolumn{4}{c}{$\lambda=1.5$}\\ \cmidrule(r){3-6} \cmidrule(r){8-11}\cmidrule(r){13-16}
MAF  & $\theta^{(0)}$ &   $e_P$        &    $e_C$       &    $e_{HL}$       &     $e_B$     &            &            $e_P$        &    $e_C$       &    $e_{HL}$       &     $e_B$    &           &      $e_P$        &    $e_C$       &    $e_{HL}$       &     $e_B$   \\ \hline
0.15 & 0                     &           &           &           &          &  &           &           &           &          &  &           &           &           &          \\
     & 1/4                   &           &           &           &          &  &           &           &           &          &  &           &           &           &          \\
     & 1/2                   &           &           &           &          &  &           &           &           &          &  &           &           &           &          \\
     & 1                     &           &           &           &          &  &           &           &           &          &  &           &           &           &          \\
     &                       &           &           &           &          &  &           &           &           &          &  &           &           &           &          \\
0.30 & 0                     &           &           &           &          &  &           &           &           &          &  &           &           &           &          \\
     & 1/4                   &           &           &           &          &  &           &           &           &          &  &           &           &           &          \\
     & 1/2                   &           &           &           &          &  &           &           &           &          &  &           &           &           &          \\
     & 1                     &           &           &           &          &  &           &           &           &          &  &           &           &           &          \\
     &                       &           &           &           &          &  &           &           &           &          &  &           &           &           &          \\
0.45 & 0                     &           &           &           &          &  &           &           &           &          &  &           &           &           &          \\
     & 1/4                   &           &           &           &          &  &           &           &           &          &  &           &           &           &          \\
     & 1/2                   &           &           &           &          &  &           &           &           &          &  &           &           &           &          \\
     & 1                     &           &           &           &          &  &           &           &           &          &  &           &           &           &          \\ \bottomrule
\end{tabular}
\end{table}


\end{document}
